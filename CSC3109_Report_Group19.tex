%%%%%%%%%%%%%%%%%%%%%%%%%%%%%%%%%%%%%%%%%
% University/School Laboratory Report
% LaTeX Template
% Version 4.0 (March 21, 2022)
%
% This template originates from:
% https://www.LaTeXTemplates.com
%
% Authors:
% Vel (vel@latextemplates.com)
% Linux and Unix Users Group at Virginia Tech Wiki
%
% License:
% CC BY-NC-SA 4.0 (https://creativecommons.org/licenses/by-nc-sa/4.0/)
%
%%%%%%%%%%%%%%%%%%%%%%%%%%%%%%%%%%%%%%%%%

%----------------------------------------------------------------------------------------
%	PACKAGES AND DOCUMENT CONFIGURATIONS
%----------------------------------------------------------------------------------------

\documentclass[
	article, % Paper size, specify a4paper (A4) or letterpaper (US letter)
	11pt, % Default font size, specify 10pt, 11pt or 12pt
	%draft, % Comment out before submission
]{CSUniSchoolLabReport}


\addbibresource{reference.bib} % Bibliography file (located in the same folder as the template)

\usepackage{enumitem}
\usepackage{hyperref}
\usepackage{import}
\usepackage{multicol}
\usepackage{authblk}

\usepackage[toc, acronym]{glossaries}
\usepackage[automake]{glossaries-extra}

%----------------------------------------------------------------------------------------
%	REPORT INFORMATION
%----------------------------------------------------------------------------------------

\title{CSC3109 Machine Learning Project Report} % Title of the report

\author[1]{Woon Jun Wei (2200624)}
\author[1]{Benjamin Loh Choon How (2201590)}
\author[1]{Low Hong Sheng Jovian (2203654)}
\author[1]{Ong Zi Xuan Max (2200717)}
\author[1]{Cleon Tay Shi Hong (2200649)}

\affil[1]{Team 19}

\date{\today} % Date of the report

\makeglossaries
\newacronym{cnn}{CNN}{Convolution Neural Network}
\newacronym{mlp}{MLP}{Multilayer Perceptron}
\newacronym{vit}{ViT}{Vision Transformer}

\newglossaryentry{Convolutional Neural Network}
{
    name={Convolutional Neural Network (CNN)},
    description={A type of neural network particularly effective for processing grid-like data, such as images, by applying convolutional operations to extract features}
}

\newglossaryentry{Multilayer Perceptron}
{
    name={Multilayer Perceptron (MLP)},
    description={A class of feedforward artificial neural network consisting of multiple layers of nodes, each layer fully connected to the next, commonly used for classification and regression tasks}
}


\newglossaryentry{Vision Transformer}
{
name={Vision Transformer (ViT)},
description={A type of neural network model that adapts the transformer architecture, originally designed for natural language processing, to image recognition tasks. ViT divides an image into fixed-size patches, linearly embeds each of them, adds positional encodings, and processes the sequence of embeddings through multiple transformer layers. This approach allows ViT to capture global interactions between patches, making it effective for large-scale image classification tasks.}
}



%----------------------------------------------------------------------------------------

\begin{document}

\maketitle % Insert the title, author and date using the information specified above

\pagebreak
\tableofcontents
\pagebreak

%%%%%%%%%%%%%%%%%%%%%%%%%%%%%%%%%%%%%%%%%%%%%%%%%
% introduction 
%%%%%%%%%%%%%%%%%%%%%%%%%%%%%%%%%%%%%%%%%%%%%%%%%
\import{Sections/}{introduction.tex}

%%%%%%%%%%%%%%%%%%%%%%%%%%%%%%%%%%%%%%%%%%%%%%%%%
% Data Preprocesing 
%%%%%%%%%%%%%%%%%%%%%%%%%%%%%%%%%%%%%%%%%%%%%%%%%
%\pagebreak
\import{Sections/}{data_preprocessing.tex}

%%%%%%%%%%%%%%%%%%%%%%%%%%%%%%%%%%%%%%%%%%%%%%%%%
% Model Training
%%%%%%%%%%%%%%%%%%%%%%%%%%%%%%%%%%%%%%%%%%%%%%%%%
%\pagebreak
\import{Sections/}{model_training.tex}

%%%%%%%%%%%%%%%%%%%%%%%%%%%%%%%%%%%%%%%%%%%%%%%%%
% Model Evaluation
%%%%%%%%%%%%%%%%%%%%%%%%%%%%%%%%%%%%%%%%%%%%%%%%%
%\pagebreak
\import{Sections/}{model_evaluation.tex}

\import{Sections/}{conclusion.tex}

% Might have use? %

% The training logs reveal the model's progression over 20 epochs. The initial accuracy of around 51.91\% in the first epoch steadily improves, reaching approximately 84.53\% by the final epoch. Similarly, the validation accuracy starts at 59.64\% and climbs to 83.93\%, indicating that the model generalizes well to unseen data. 

% % Overall performance of mlp %
% The MLP model demonstrates strong performance on the test dataset, achieving an accuracy of 83.9\%. This accuracy reflects the model's ability to make correct predictions on unseen data. The precision, recall, and F1-score metrics provide a detailed assessment of the model's performance for each class. For class 0, the precision is 0.80, indicating that 80\% of the samples predicted as class 0 were correct, while the recall is 0.91, indicating that 91\% of the actual class 0 samples were identified correctly. Similarly, for class 1, the precision is 0.89 and the recall is 0.77. These metrics suggest that the model performs slightly better in identifying class 1 compared to class 0. The F1-scores for both classes are around 0.84, indicating a good balance between precision and recall. The classification report also includes macro and weighted average values, which provide a summarized view of the model's performance across both classes. The macro average values are 0.84 for all metrics, while the weighted average values are slightly higher at 0.85, reflecting a slight bias towards the slightly larger class (class 1). Overall, the classification report suggests that the MLP model generalizes well to unseen data and avoids overfitting, making it effective for the NLOS/LOS classification task.

% \begin{center}
% 	\begin{tabular}{l r}
% 		Date Performed: & February 13, 2022 \\ % Date the experiment was performed
% 		Partners: & Cecilia \textsc{Smith} \\ % Partner names
% 		& Tajel \textsc{Khumalo} \\
% 		Instructor: & Professor \textsc{Rivera} % Instructor/supervisor
% 	\end{tabular}
% \end{center}

% If you need to include an abstract, uncomment the lines below
%\begin{abstract}
%	Abstract text
%\end{abstract}

%----------------------------------------------------------------------------------------
%	OBJECTIVE
%----------------------------------------------------------------------------------------

% \section{Objective}





% To determine the atomic weight of magnesium via its reaction with oxygen and to study the stoichiometry of the reaction (as defined in \ref{definitions}):

% \begin{center}
% 	\ce{2 Mg + O2 -> 2 MgO} % Chemical equations entered in \ce{} commands, see the mhchem package documentation for more information
% \end{center}

% If you have more than one objective, uncomment the below:
%\begin{description}
%	\item[First Objective] \hfill \\
%	Objective 1 text
%	\item[Second Objective] \hfill \\
%	Objective 2 text
%\end{description}

% \begin{description}
% 	\item[Stoichiometry] The relationship between the relative quantities of substances taking part in a reaction or forming a compound, typically a ratio of whole integers.
% 	\item[Atomic mass] The mass of an atom of a chemical element expressed in atomic mass units. It is approximately equivalent to the number of protons and neutrons in the atom (the mass number) or to the average number allowing for the relative abundances of different isotopes. 
% \end{description} 
 
%----------------------------------------------------------------------------------------
%	EXPERIMENTAL DATA
%----------------------------------------------------------------------------------------

% \section{Experimental Data}

% \begin{tabular}{l l}
% 	Mass of empty crucible & \SI{7.28}{\gram}\\ % Scientific/technical units are output using the \SI command, see the siunitx package documentation for more information on how to use this command
% 	Mass of crucible and magnesium before heating & \SI{8.59}{\gram}\\
% 	Mass of crucible and magnesium oxide after heating & \SI{9.46}{\gram}\\
% 	Balance used & \#4\\
% 	Magnesium from sample bottle & \#1
% \end{tabular}

%----------------------------------------------------------------------------------------
%	SAMPLE CALCULATION
%----------------------------------------------------------------------------------------

% \section{Sample Calculation}

% \begin{tabular}{ll}
% 	Mass of magnesium metal & = \SI{8.59}{\gram} - \SI{7.28}{\gram}\\
% 	& = \SI{1.31}{\gram}\\
% 	Mass of magnesium oxide & = \SI{9.46}{\gram} - \SI{7.28}{\gram}\\
% 	& = \SI{2.18}{\gram}\\
% 	Mass of oxygen & = \SI{2.18}{\gram} - \SI{1.31}{\gram}\\
% 	& = \SI{0.87}{\gram}
% \end{tabular}

% Because of this reaction, the required ratio is the atomic weight of magnesium: \SI{16.00}{\gram} of oxygen as experimental mass of Mg: experimental mass of oxygen or $\frac{x}{1.31} = \frac{16}{0.87}$ from which, $M_{\ce{Mg}} = 16.00 \times \frac{1.31}{0.87} = 24.1 = \SI{24}{\gram\per\mole}$ (to two significant figures).

%----------------------------------------------------------------------------------------
%	RESULTS AND CONCLUSIONS
%----------------------------------------------------------------------------------------

% \section{Results and Conclusions}

% The atomic weight of magnesium is concluded to be \SI{24}{\gram\per\mol}, as determined by the stoichiometry of its chemical combination with oxygen. This result is in agreement with the accepted value.

% \begin{figure}[H] % [H] forces the figure to be placed exactly where it appears in the text
% 	\centering % Horizontally center the figure
% 	\includegraphics[width=0.65\textwidth]{placeholder} % Include the figure
% 	\caption{Figure caption.}
% \end{figure}

%----------------------------------------------------------------------------------------
%	DISCUSSION
%----------------------------------------------------------------------------------------

% \section{Discussion of Experimental Uncertainty}

% The accepted value (periodic table) is \SI{24.3}{\gram\per\mole} \autocite{Smith:2022qr}. The percentage discrepancy between the accepted value and the result obtained here is 1.3\%. Because only a single measurement was made, it is not possible to calculate an estimated standard deviation (see \textcite{Smith:2021jd}).
 
% The most obvious source of experimental uncertainty is the limited precision of the balance. Other potential sources of experimental uncertainty are: the reaction might not be complete; if not enough time was allowed for total oxidation, less than complete oxidation of the magnesium might have, in part, reacted with nitrogen in the air (incorrect reaction); the magnesium oxide might have absorbed water from the air, and thus weigh ``too much." Because the result obtained is close to the accepted value it is possible that some of these experimental uncertainties have fortuitously cancelled one another.

%----------------------------------------------------------------------------------------
%	ANSWERS TO DEFINITIONS
%----------------------------------------------------------------------------------------

% \section{Answers to Definitions}

% \begin{enumerate}
% 	\item The \textit{atomic weight of an element} is the relative weight of one of its atoms compared to C-12 with a weight of 12.0000000$\ldots$, hydrogen with a weight of 1.008, to oxygen with a weight of 16.00. Atomic weight is also the average weight of all the atoms of that element as they occur in nature.
% 	\item The \textit{units of atomic weight} are two-fold, with an identical numerical value. They are g/mole of atoms (or just g/mol) or amu/atom.
% 	\item \textit{Percentage discrepancy} between an accepted (literature) value and an experimental value is:
% 		\begin{equation*}
% 			\frac{\mathrm{experimental\;result} - \mathrm{accepted\;result}}{\mathrm{accepted\;result}}
% 		\end{equation*}
% \end{enumerate}

%----------------------------------------------------------------------------------------
%	BIBLIOGRAPHY
%----------------------------------------------------------------------------------------

\printbibliography % Output the bibliography

%----------------------------------------------------------------------------------------

%%%%%%%%%%%%%%%%%%%%%%%%%%%%%%%%%%%%%%%%%%%%%%%%%
% Glossary
%%%%%%%%%%%%%%%%%%%%%%%%%%%%%%%%%%%%%%%%%%%%%%%%%
\pagebreak

\printglossary[type=\acronymtype]

\printglossary

%%%%%%%%%%%%%%%%%%%%%%%%%%%%%%%%%%%%%%%%%%%%%%%%%
% Contributions
%%%%%%%%%%%%%%%%%%%%%%%%%%%%%%%%%%%%%%%%%%%%%%%%%
%\pagebreak
% \import{Sections/}{contributions.tex}


\end{document}
