\section{U-Net Appendix}\label{s:unetAppendix}

\begin{figure}[H]
  \begin{center}
    \includegraphics[width=0.35\textwidth]{unet/unetpp_model2.png}
  \end{center}
  \caption{U-Net++ model architecture}\label{fig:unetpp_model}
\end{figure}

Where \texttt{model\_12} is the U-Net++ model with EfficientNetB1 encoder. The Output Layer of \texttt{model\_12} is also used for the segmentation attempt to show what the model deems as the important parts of the image for the classification task.

%\includepdf[
%    %% Include all pages of the PDF
%    pages=-,
%    %% make this page have the usual page style
%    %% (you can change it to plain etc). By default pdfpages
%    %% sets the pagecommand to \pagestyle{empty}
%    pagecommand={\pagestyle{headings}},  
%    %% Add a "section" entry to the ToC with the heading
%    %% "Quilling Shapes" and the label "sec:shapes"
%    %addtotoc={1,section,1,Quilling Shapes,sec:shapes}
%    ]
%%% The pdf file itself
%{unet.pdf}
