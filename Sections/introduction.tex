\section{Introduction}\label{s:introduction}
Machine learning (ML) has emerged as a pivotal technology, driving advancements across various fields, including healthcare, climate science, and energy management. With the advent of deep learning, a subset of ML, machines are now capable of achieving human-like perception and decision-making abilities. This project aims to leverage deep learning techniques to address real-world challenges, particularly in the realm of healthcare. By focusing on the classification of brain MRI images, this project provides an opportunity to apply theoretical knowledge to practical problems, thereby enhancing our understanding of advanced ML methodologies and their applications.

\section{Problem Statement}\label{s:problem-statement}
The primary problem addressed in this project is the accurate classification of brain MRI images into four distinct categories. Brain MRI scans are crucial in diagnosing and monitoring various neurological conditions. Accurate classification of these images can significantly aid medical professionals in early detection and treatment planning. However, due to the complexity and variability in MRI images, developing robust classification models poses a significant challenge. This project seeks to develop, evaluate, and compare multiple deep learning models to achieve high accuracy in classifying these MRI images, ultimately contributing to improved healthcare outcomes.

\section{Objective}\label{s:objective}
The overall objective of this project is to apply and extend deep learning techniques to solve a specific real-world problem in healthcare: the classification of brain MRI images. 
% The specific objectives include:
% \begin{itemize}
%     \item To develop a comprehensive understanding of the challenges associated with classifying brain MRI images.
%     \item To perform exploratory data analysis and data preprocessing to prepare the dataset for model training.
%     \item To design, implement, and evaluate at least three distinct deep learning models for the classification task.
%     \item To conduct a thorough performance evaluation of the models, highlighting their strengths and weaknesses.
%     \item To document the entire process, providing clear motivations, methodologies, and insights derived from the results.
%     \item To enhance and customize existing open-source machine learning libraries, demonstrating an in-depth understanding and innovative application of these tools.
% \end{itemize}
% This project not only aims to achieve technical proficiency in deep learning but also to foster a constructive attitude towards tackling large-scale ML problems, thereby preparing students for future industry demands.
