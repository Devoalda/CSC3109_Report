
\section{Introduction}\label{s:introduction}

Machine learning (ML) has become a transformative technology, driving innovation across numerous fields such as healthcare\cite{nazar_systematic_2021}, healthcare research \cite{doupe_machine_2019} and climate science\cite{rolnick_tackling_2022}. Within ML, deep learning has emerged as a particularly powerful subset, enabling machines to perform tasks that require human-like perception and decision-making. This project aims to harness deep learning techniques to tackle real-world challenges in healthcare, specifically focusing on the classification of brain MRI images. By applying deep learning methodologies to this task, the project bridges theoretical concepts with practical applications, thereby advancing our understanding and capabilities in advanced ML techniques.

%\section{Problem Statement}\label{s:problem-statement}

% Why is there a need for brain MRI and classification?
Brain magnetic resonance imaging (MRI) is a non-invasive imaging technique that plays a crucial role in diagnosing and monitoring various brain conditions, including tumors. The interpretation of brain MRI images is a complex and time-consuming process, requiring expert knowledge and experience. Accurate classification of brain MRI images is essential for timely diagnosis and treatment planning, as different types of brain tumors have distinct characteristics and treatment strategies. Patients may receive imaging studies as part of their routine care or to investigate specific symptoms, such as headaches, seizures, or cognitive changes \cite{lapointe_primary_2018}. In many cases, the radiologist's report on the MRI findings is critical for guiding clinical decisions and patient management.

Accurate classification of brain MRI images is a significant challenge in the medical field. Brain MRI scans are essential for diagnosing and monitoring various neurological conditions, yet the complexity and variability of these images make classification difficult. Misclassifications can lead to delays in diagnosis and treatment, impacting patient outcomes\cite{iorgulescu_misclassification_2019}. Thus, there is a critical need for robust classification models that can reliably distinguish between different types of brain tumors.

As the availability of MRI data increases, the demand for automated tools to assist radiologists in interpreting these images grows. Current manual methods are time-consuming and prone to human error, highlighting the need for efficient, automated solutions \cite{lenchik_automated_2019}. This project seeks to address this problem by developing, evaluating, and comparing multiple deep learning models to achieve high accuracy in classifying brain MRI images. By leveraging deep learning techniques, the project aims to reduce the workload on healthcare professionals, enhance diagnostic efficiency, and improve patient care through early detection and timely medical interventions.

%\section{Objective}\label{s:objective}

The overarching objective of this project is to develop deep learning models that aid radiologists and doctors in the efficient diagnosis of brain tumors, facilitating early intervention and improving patient outcomes. By applying and extending advanced deep learning techniques, this project aims to produce models capable of accurately classifying brain MRI images. These models will support healthcare professionals by reducing diagnostic time and enhancing the accuracy of diagnoses, ultimately contributing to more timely and effective medical interventions.

