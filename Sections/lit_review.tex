\section{Literature Review}\label{s:lit_review}

% joviana check plox %
Deep learning has emerged as a pivotal technology in the medical field, particularly in the classification of brain MRI images. Various deep learning methodologies have been explored to improve the accuracy and efficiency of diagnosing brain tumors from MRI scans. This literature review delves into the recent advancements and approaches in this domain, highlighting the significant contributions and methodologies that have been employed.

Convolutional Neural Networks (CNNs) have been extensively used for brain MRI classification. A notable approach is the use of ResNet-18 architecture for differentiating MRI sequence types, achieving an impressive accuracy of 97.9\% on test sets. This model demonstrates the capability of CNNs in handling complex MRI data efficiently\cite{doi:10.1148/ryai.230095}.

Transfer learning has become a prominent technique in enhancing the performance of deep learning models, especially when dealing with limited labeled data. This approach involves leveraging pre-trained models on large datasets and fine-tuning them on specific tasks such as brain MRI classification.

For example, transfer learning architectures such as InceptionV3, VGG19, DenseNet121, and MobileNet have been applied in brain MRI classification for brain tumor diagnosis, with MobileNet demonstrating the highest accuracy of 99.60\% \cite{Islam_Barua_Rahman_Ahammed_Akter_Uddin_2023}. This highlights the effectiveness of using pre-trained models in achieving high accuracy with relatively small datasets.

In another study, transfer learning was utilized to address the challenges of small sample sizes and high dimensionality in whole-brain fMRI data. The approach showed improved performance compared to training models from scratch, demonstrating the benefits of transfer learning in enhancing model performance with limited data \cite{10.1007/978-3-030-32695-1_7}.

Furthermore, a study on brain tumor detection employed transfer learning by fine-tuning a deep learning model trained on a large dataset of natural images. This method resulted in comparable performance even with a smaller dataset, showcasing the practicality and effectiveness of transfer learning in medical imaging \cite{10125766}.

In conclusion, deep learning has significantly advanced the field of brain MRI classification, offering robust, accurate, and efficient solutions for diagnosing and monitoring brain tumors. The continuous evolution of CNN architectures, transfer learning techniques, optimization strategies, and preprocessing methods promises further enhancements in the accuracy and reliability of these models, ultimately contributing to improved patient outcomes and reduced diagnostic times. These advancements not only highlight the transformative potential of deep learning in medical imaging but also underscore the importance of ongoing research and development. By continuing to refine and innovate upon these techniques, the medical community can look forward to even more effective and efficient diagnostic tools, paving the way for better patient care and outcomes.