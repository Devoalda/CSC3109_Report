\section{Reflections}\label{s:reflections}

\subsection{Woon Jun Wei (2200624)}

Throughout this brain tumor classification project, I have engaged in a comprehensive exploration of deep learning models and techniques. The primary objective was to classify brain images using various deep learning architectures, including UNet, Inception, VGG16, and Vision Transformers (ViT). Additionally, I experimented with data augmentation methods, including the potential use of Generative Adversarial Networks (GANs), and reviewed literature on affine augmentations to understand their benefits.

During the project, I applied the insights gained from my literature review to practical tasks such as data exploration, preprocessing, and augmentation. Leveraging tools such as TensorFlow and Keras, I selected and architected models, fine-tuning them with the help of Optuna for hyperparameter optimization. This iterative process of model selection and tuning was guided by referencing various GitHub repositories and research papers, ensuring the integration of state-of-the-art techniques and best practices.

One of the significant learning outcomes of this project was gaining in-depth knowledge about MRIs and brain tumors, and understanding the necessity of automated processes in medical imaging. This knowledge was crucial in appreciating the context and implications of the classification task. Observing performance metrics provided valuable insights into model performance, highlighting the need for more extensive datasets to improve generalization and robustness of the models.

Moreover, I learned advanced techniques for image segmentation, utilizing features extracted through deep learning to enhance classification accuracy. This aspect of the project not only improved my technical skills but also underscored the importance of feature extraction in medical image analysis. The integration of segmentation and classification processes demonstrated the potential for more accurate and efficient diagnostic tools.

In summary, this project provided a valuable opportunity to integrate theoretical knowledge with practical application. It culminated in a deeper understanding of both the challenges and potential solutions in automated brain tumor classification. The experience emphasized the importance of continuous learning and adaptation in the rapidly evolving field of deep learning and medical imaging, reinforcing the need for ongoing research and development to address complex medical challenges.

\subsection{Benjamin Loh Choon How (2201590)}

\subsection{Low Hong Sheng Jovian (2203654)}

\subsection{Ong Zi Xuan Max (2200717)}

\subsection{Cleon Tay Shi Hong (2200649)}
