\section{Reflections}\label{s:reflections}

\subsection{Woon Jun Wei (2200624)}

Throughout this brain tumor classification project, I have engaged in a comprehensive exploration of deep learning models and techniques. The primary objective was to classify brain images using various deep learning architectures, including UNet, Inception, VGG16, and Vision Transformers (ViT). Additionally, I experimented with data augmentation methods, including the potential use of Generative Adversarial Networks (GANs), and reviewed literature on affine augmentations to understand their benefits.

During the project, I applied the insights gained from my literature review to practical tasks such as data exploration, preprocessing, and augmentation. Leveraging tools such as TensorFlow and Keras, I selected and architected models, fine-tuning them with the help of Optuna for hyperparameter optimization. This iterative process of model selection and tuning was guided by referencing various GitHub repositories and research papers, ensuring the integration of state-of-the-art techniques and best practices.

One of the significant learning outcomes of this project was gaining in-depth knowledge about MRIs and brain tumors, and understanding the necessity of automated processes in medical imaging. This knowledge was crucial in appreciating the context and implications of the classification task. Observing performance metrics provided valuable insights into model performance, highlighting the need for more extensive datasets to improve generalization and robustness of the models.

Moreover, I learned advanced techniques for image segmentation, utilizing features extracted through deep learning to enhance classification accuracy. This aspect of the project not only improved my technical skills but also underscored the importance of feature extraction in medical image analysis. The integration of segmentation and classification processes demonstrated the potential for more accurate and efficient diagnostic tools.

In summary, this project provided a valuable opportunity to integrate theoretical knowledge with practical application. It culminated in a deeper understanding of both the challenges and potential solutions in automated brain tumor classification. The experience emphasized the importance of continuous learning and adaptation in the rapidly evolving field of deep learning and medical imaging, reinforcing the need for ongoing research and development to address complex medical challenges.

\subsection{Benjamin Loh Choon How (2201590)}

Throughout this project, where I trained a deep learning model to classify brain MRI images, I have gained a deeper understanding of the broader implications of applying machine learning in medical imaging. This journey has helped me appreciate the complexities and benefits of using advanced technology to improve healthcare. To tackle the problem statement, I started exploring multiple pre-trained models I could use for transfer learning due to the small amount of data we were given to train our model, and I chose ResNet-50 as my model after reading multiple research papers which recorded good performances for tasks like classifying brain MRI images. I started off by carefully preparing the data after learning useful preprocessing techniques referenced from the research papers I have read. Using tools like OpenCV and Python, I converted images to grayscale to make them simpler and blurred them to remove noise. I then used techniques to highlight the brain and remove unnecessary background. This careful preparation made sure the data was clear and focused, which is important for making the model work better.

Once the data was ready, I moved on to setting up the model for training. I implemented ResNet-50 using TensorFlow and Keras, which are powerful tools for deep learning. I utilized data augmentation techniques to increase the diversity of the training data, such as flipping, rotating, and shifting the images. These methods were crucial for enhancing the model’s ability to generalize from the limited dataset. The process of hyperparameter tuning was guided by the Optuna framework, which allowed me to fine-tune the model's settings systematically and efficiently. This iterative process helped me achieve the best possible model performance by experimenting with different configurations and selecting the most effective parameters.

During training, I focused on strategies to ensure that the model learned effectively and avoided overfitting. I used ModelCheckpoint to save the best version of the model based on validation performance and EarlyStopping to halt training if no improvement was observed, which prevented unnecessary training time and potential overfitting. The training process was monitored carefully to maintain a balance between model complexity and performance. I also applied K-Folds cross-validation to evaluate the model's generalization capabilities, providing a robust estimate of how well the model would perform on unseen data.

Reflecting on this experience, I have come to understand the critical role of machine learning in advancing medical diagnostics. The knowledge I gained about the importance of accurate and reliable MRI image classification has highlighted the potential impact of these technologies on patient care and medical decision-making. It reinforced the need for ongoing research and development in this field to continue improving diagnostic tools and outcomes for patients.

Overall, this project has been a valuable learning journey that has equipped me with practical skills and insights into machine learning and medical imaging. It has demonstrated the importance of meticulous data preparation, thoughtful model selection, and continuous optimization. The experience has solidified my ability to develop sophisticated machine learning solutions for healthcare challenges and has emphasized the significance of innovation and precision in the ever-evolving landscape of technology and healthcare. This project not only expanded my technical expertise but also deepened my commitment to using advanced technologies to contribute to better healthcare outcomes.

\subsection{Low Hong Sheng Jovian (2203654)}

\subsection{Ong Zi Xuan Max (2200717)}

\subsection{Cleon Tay Shi Hong (2200649)}
