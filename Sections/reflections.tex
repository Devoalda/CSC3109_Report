\section{Reflections}\label{s:reflections}

\subsection{Woon Jun Wei (2200624)}

Throughout this brain tumor classification project, I have engaged in a comprehensive exploration of deep learning models and techniques. The primary objective was to classify brain images using various deep learning architectures, including UNet, Inception, VGG16, and Vision Transformers (ViT). Additionally, I experimented with data augmentation methods, including the potential use of Generative Adversarial Networks (GANs), and reviewed literature on affine augmentations to understand their benefits.

During the project, I applied the insights gained from my literature review to practical tasks such as data exploration, preprocessing, and augmentation. Leveraging tools such as TensorFlow and Keras, I selected and architected models, fine-tuning them with the help of Optuna for hyperparameter optimization. This iterative process of model selection and tuning was guided by referencing various GitHub repositories and research papers, ensuring the integration of state-of-the-art techniques and best practices.

One of the significant learning outcomes of this project was gaining in-depth knowledge about MRIs and brain tumors, and understanding the necessity of automated processes in medical imaging. This knowledge was crucial in appreciating the context and implications of the classification task. Observing performance metrics provided valuable insights into model performance, highlighting the need for more extensive datasets to improve generalization and robustness of the models.

Moreover, I learned advanced techniques for image segmentation, utilizing features extracted through deep learning to enhance classification accuracy. This aspect of the project not only improved my technical skills but also underscored the importance of feature extraction in medical image analysis. The integration of segmentation and classification processes demonstrated the potential for more accurate and efficient diagnostic tools.

I dedicated substantial time and effort to researching and understanding the metrics used for evaluating model performance. Through reviewing research and academic papers, I identified and explored several key metrics, including Dice Similarity Coefficient (DSC), Sensitivity, Specificity, and Accuracy, referenced from Imtiaz et al. \cite{imtiaz_brain_2023}. These metrics provided valuable insights into my model's performance, helping to identify strengths and areas for improvement. Additionally, I examined other relevant metrics for segmentation tasks, such as Intersection over Union (IoU), which offered further nuances in evaluating the segmentation quality.

Furthermore, I explored the potential application of my models, such as UNet++ with EfficientNet (Eff-UNet), in other medical domains, including pneumonia detection through X-rays and other imaging tasks. This exploration highlighted the versatility and adaptability of the models developed, underscoring their potential impact across various medical imaging challenges.

All in all, this project provided a valuable opportunity to integrate theoretical knowledge with practical application. It culminated in a deeper understanding of both the challenges and potential solutions in automated brain tumor classification. The experience emphasized the importance of continuous learning and adaptation in the rapidly evolving field of deep learning and medical imaging, reinforcing the need for ongoing research and development to address complex medical challenges.

\subsection{Benjamin Loh Choon How (2201590)}

Throughout this project, where I trained a deep learning model to classify brain MRI images, I have gained a deeper understanding of the broader implications of applying machine learning in medical imaging. This journey has helped me appreciate the complexities and benefits of using advanced technology to improve healthcare. To tackle the problem statement, I started exploring multiple pre-trained models I could use for transfer learning due to the small amount of data we were given to train our model, and I chose ResNet-50 as my model after reading multiple research papers which recorded good performances for tasks like classifying brain MRI images. I started off by carefully preparing the data after learning useful preprocessing techniques referenced from the research papers I have read. Using tools like OpenCV and Python, I converted images to grayscale to make them simpler and blurred them to remove noise. I then used techniques to highlight the brain and remove unnecessary background. This careful preparation made sure the data was clear and focused, which is important for making the model work better.

Once the data was ready, I moved on to setting up the model for training. I implemented ResNet-50 using TensorFlow and Keras, which are powerful tools for deep learning. I utilized data augmentation techniques to increase the diversity of the training data, such as flipping, rotating, and shifting the images. These methods were crucial for enhancing the model’s ability to generalize from the limited dataset. The process of hyperparameter tuning was guided by the Optuna framework, which allowed me to fine-tune the model's settings systematically and efficiently. This iterative process helped me achieve the best possible model performance by experimenting with different configurations and selecting the most effective parameters.

During training, I focused on strategies to ensure that the model learned effectively and avoided overfitting. I used ModelCheckpoint to save the best version of the model based on validation performance and EarlyStopping to halt training if no improvement was observed, which prevented unnecessary training time and potential overfitting. The training process was monitored carefully to maintain a balance between model complexity and performance. I also applied K-Folds cross-validation to evaluate the model's generalization capabilities, providing a robust estimate of how well the model would perform on unseen data.

Reflecting on this experience, I have come to understand the critical role of machine learning in advancing medical diagnostics. The knowledge I gained about the importance of accurate and reliable MRI image classification has highlighted the potential impact of these technologies on patient care and medical decision-making. It reinforced the need for ongoing research and development in this field to continue improving diagnostic tools and outcomes for patients.

Overall, this project has been a valuable learning journey that has equipped me with practical skills and insights into machine learning and medical imaging. It has demonstrated the importance of meticulous data preparation, thoughtful model selection, and continuous optimization. The experience has solidified my ability to develop sophisticated machine learning solutions for healthcare challenges and has emphasized the significance of innovation and precision in the ever-evolving landscape of technology and healthcare. This project not only expanded my technical expertise but also deepened my commitment to using advanced technologies to contribute to better healthcare outcomes.


\subsection{Low Hong Sheng Jovian (2203654)}

As a computer science student, working on a machine learning project to classify brain MRI images has been an enriching experience. This project allowed me to apply theoretical knowledge to a real-world problem with significant medical implications. I faced the initial challenge of understanding the complexities of medical imaging and brain MRI classification, which required deep engagement with medical literature and advanced machine learning models like CNNs and Vision Transformers (ViTs).

One of the most rewarding aspects was the practical application of classroom concepts. Implementing and experimenting with convolutional layers, self-attention mechanisms, overfitting, and cross-validation provided a deeper, more intuitive grasp of these theories. Overcoming computational limitations, even with resources like Google Colab Pro, was a significant part of the project, necessitating innovative solutions to balance computational efficiency and model performance.

A particularly enlightening part of the project was researching and experimenting with Vision Transformers (ViTs). Initially designed for natural language processing, ViTs presented a novel approach to image classification through their self-attention mechanism. Diving into the intricacies of ViTs, understanding their architecture, and experimenting with them firsthand was incredibly satisfying. This hands-on experimentation allowed me to witness the model's capabilities and limitations, particularly in handling raw images without extensive preprocessing.

Throughout the project, I gained valuable insights into the strengths and weaknesses of different deep learning models. CNNs handled spatial hierarchies well, while ViTs excelled in dynamically focusing on relevant image segments. These insights were critical in selecting appropriate models for various tasks. The project also highlighted the potential impact of machine learning in medical diagnostics, emphasizing the need for larger, diverse datasets to improve model generalization and reliability.

However, the journey was not without its difficulties. Coming into this project with no prior experience in machine learning, I often found myself overwhelmed by the vastness of the field. It was challenging to understand the many complex concepts and algorithms, and I frequently had to rely on extensive research just to have a clue about what was going on. Determining the right direction to move in was often daunting and required a lot of trial and error.

This experience has reinforced my passion for machine learning and its transformative potential. It has equipped me with the skills to tackle more complex problems and motivated me to explore advanced techniques like data augmentation and transfer learning. Looking ahead, if I have sufficient computational power, I am eager to explore the potential of segmentation with ViT models, particularly by splitting the dataset into patch sizes of 16. This future direction could offer deeper insights and improvements in medical image analysis, further advancing the capabilities of machine learning in healthcare.

This project has been a crucial part of my academic journey, greatly shaping my skills and aspirations as a computer science student. It has deepened my understanding of machine learning and its applications, especially in healthcare. Working on a real-world problem has improved my analytical and problem-solving skills, teaching me perseverance and adaptability.

The project emphasized the importance of continuous learning and staying updated with technology advancements. It inspired me to pursue further studies in machine learning, focusing on medical imaging and diagnostics. Collaborating with peers and receiving guidance from mentors provided diverse perspectives and improved my teamwork skills.

Overall, this experience has given me a strong foundation in machine learning, practical skills in modern frameworks, and an understanding of the field's practical applications. It has prepared me for future endeavors, driving me to contribute to innovative solutions that solve real-world problems.


\subsection{Ong Zi Xuan Max (2200717)}

As an individual with a fundamental knowledge of neural networks from a previous university module, this brain tumor MRI image classification project has provided an opportunity to gain a more in-depth comprehension of the applications of machine learning. The goal was to develop various deep learning neural network models as a team that could classify different classes of brain tumor MRI images. To achieve this goal, extensive research and understanding of various applications in each stage of building a specific model architecture were required, allowing me to expand and deepen my knowledge in machine learning.

Putting to work the insights gained from analyzing multiple research papers and online articles, I applied them to practical tasks such as tailoring augmentation and distributing data to the selected pre-trained Xception model architecture. Continuous reading of framework and library documentation aided in understanding how to create suitable tools and solutions for various phases of building the model. This included creating and modifying the architecture of the pre-trained model, implementing various callbacks to train the model effectively, and using techniques to prevent underfitting and overfitting while evaluating the model after each trial.

Fine-tuning the model was a significant phase, characterized by a repetitive cycle of reading, comprehending, tuning, and evaluation. Leveraging Optuna for hyperparameter optimization helped reduce a considerable amount of time in manual tuning and provided reliable results. It was not just a plug-and-play framework but required grasping the specifics of each execution line written.

Reflecting on this experience, I have come to appreciate the complexities and potential of applying advanced machine learning techniques to real-world medical imaging problems. The project reinforced the importance of meticulous data preparation, thoughtful model selection, and continuous optimization. It has equipped me with practical skills and insights, deepening my knowledge and widening my exposure to using advanced technologies in tailoring deep neural network models for specific use cases.

\subsection{Cleon Tay Shi Hong (2200649)}

Through this brain tumor classification project, I gained the opportunity to get a deeper understanding of machine learning and its applications. The objective of this project is to research and develop a model each for classifying brain tumor MRI images, then finally decide on the best to adapt it to the project. To move forward, we did many research and try to understand more model architecture and how it could be implemented and adapted into our project.

After researching on a few models by reading on research paper and articles written by the other developers, I chose the pre-trained DenseNet121 model architecture. Using this pre-trained model, I modify and customized it to fit our objective needs. Continuously reading and understanding how to develop this model architecture from the different phases and layers. This involved in designing and adjusting of the pre-trained model, and deploying methods to prevent overfitting and underfitting while evaluating the result for each iterations.

Looking back on this experience in developing and customising a pre-trained model to classify brain tumor MRI images for the medical industrial uses. I gained deeper appreciation for the potential application of the machine learning methods in the medical industry. After completing this assignment I feel more capable of handling such application that could help further improve the current technology used in the medical industry.

Overall, this project has given me a valuable learning experience, allowing me to practice my practical skills and insights into machine learning. It highlighted on how important data preparation, model selection and optimization is. This experience has up skilled my ability to develop more complex machine learning programs for the healthcare industry. Also pointed out the importance of innovating in the field of the technology an healthcare. This project not only deepened my technical expertise, but also enhanced my commitment to push the advance technology to another level to improve healthcare outcome.
