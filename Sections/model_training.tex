\section{Data Mining}\label{data_mining}

In this section, various data mining techniques are employed to extract meaningful insights from the preprocessed data.

\subsection{Overview}\label{overview}

All models were trained on Google's Colab plaform, utilizing the GPU runtime for faster training. In the following subsections, we will discuss the data mining techniques used to model the brain tumor classification task. The techniques include data augmentation, data splitting, and model selection.

\subsection{Model Selection}\label{model_selection}

Model selection is a critical step in developing an effective machine learning model, especially for complex tasks like brain tumor classification. The choice of model impacts the accuracy, efficiency, and overall performance of the solution. This section discusses the use of transfer learning with pretrained models such as VGG16 and ResNet50, as well as deep learning approaches like U-Net, which have demonstrated high performance in related tasks such as the BraTS Competition.

The BraTS (Brain Tumor Segmentation) Competition has established itself as a benchmark for brain tumor segmentation tasks over several years. The competition has highlighted various models and techniques that consistently achieve superior results. Among the most commonly used models by top teams in the competition are U-Net, VGG16, and EfficientNet. These models have been chosen due to their proven ability to effectively handle medical imaging tasks, making them ideal candidates for our project.

\paragraph{Transfer Learning with Pretrained Models:}
Transfer learning involves leveraging pretrained models, which have already been trained on large datasets, and fine-tuning them for specific tasks. This approach is particularly useful when dealing with small datasets, as it allows the model to benefit from the knowledge gained during pretraining. 

\textbf{VGG16:} VGG16 is a convolutional neural network (CNN) known for its simplicity and effectiveness. It has been pretrained on the ImageNet dataset, which contains millions of images across thousands of categories. For our brain tumor classification task, VGG16 can be fine-tuned to learn the specific features of MRI images, leveraging its deep architecture to capture intricate details.

\textbf{ResNet50:} ResNet50 is another pretrained model that has gained popularity for its performance and efficiency. It introduces a residual learning framework to ease the training of deep networks. ResNet50's architecture allows it to maintain performance while being computationally efficient, making it a suitable choice for our dataset.

\paragraph{Deep Learning with U-Net:}
U-Net is a deep learning model specifically designed for biomedical image segmentation. Its architecture consists of a contracting path to capture context and a symmetric expanding path for precise localization. U-Net has been widely used in medical imaging due to its ability to segment images accurately. In the context of brain tumor classification, U-Net can be adapted to highlight and classify different regions of MRI scans, making it a powerful tool for our project.

\paragraph{Model Selection Process:}
The model selection process involves evaluating the suitability of each model based on the dataset characteristics and the specific requirements of the classification task. Given the relatively small size of our dataset (dataset\_19), models that can generalize well from limited data are preferred. 

% \begin{enumerate}
%   \item \textbf{VGG16} is selected for its deep architecture and proven performance in image classification tasks.
%   \item \textbf{ResNet50} is chosen for its balance of high accuracy and computational efficiency.
%   \item \textbf{U-Net} is included due to its specialized design for biomedical image segmentation, which can be adapted for classification purposes.
% \end{enumerate}

By leveraging these models, we aim to achieve high accuracy in classifying brain MRI images, thereby supporting healthcare professionals in the efficient diagnosis of brain tumors and facilitating early intervention. The combination of transfer learning and deep learning techniques ensures that our approach is both robust and effective, despite the limited size of the dataset.


%%%%%%%%%%%%%%%%%%%%%%%%%%%%%%%%%%%%%%%%%%%%%%%%%
% Individual Model SUBSECTIONS
%%%%%%%%%%%%%%%%%%%%%%%%%%%%%%%%%%%%%%%%%%%%%%%%%
\import{Sections/models/}{unet.tex}

\import{Sections/models/}{vgg16.tex}

\import{Sections/models/}{resnet50.tex}
