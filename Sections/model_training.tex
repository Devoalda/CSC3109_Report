\section{Data Mining and Evaluation}\label{data_mining}

In this section, various data mining techniques are employed to extract meaningful insights from the preprocessed data.

\subsection{Overview}\label{overview}

All models were trained on Google's Colab plaform, utilizing the GPU runtime for faster training. In the following subsections, we will discuss the data mining techniques used to model the brain tumor classification task. The techniques include data augmentation, data splitting, and model selection.

\subsection{Model Selection}\label{model_selection}

Model selection is a critical step in developing an effective machine learning model, especially for complex tasks like brain tumor classification. The choice of model impacts the accuracy, efficiency, and overall performance of the solution. This section discusses the use of transfer learning with pretrained models such as VGG16 and ResNet50, as well as deep learning approaches like U-Net, which have demonstrated high performance in related tasks such as the BraTS Competition.

The BraTS (Brain Tumor Segmentation) Competition has established itself as a benchmark for brain tumor segmentation tasks over several years. The competition has highlighted various models and techniques that consistently achieve superior results. Among the most commonly used models by top teams in the competition are U-Net, VGG16, and EfficientNet. These models have been chosen due to their proven ability to effectively handle medical imaging tasks, making them ideal candidates for our project.

\paragraph{Transfer Learning with Pretrained Models:}
Transfer learning involves leveraging pretrained models, which have already been trained on large datasets, and fine-tuning them for specific tasks. This approach is particularly useful when dealing with small datasets, as it allows the model to benefit from the knowledge gained during pretraining. 

\textbf{VGG16:} VGG16 is a convolutional neural network (CNN) known for its simplicity and effectiveness. It has been pretrained on the ImageNet dataset, which contains millions of images across thousands of categories. For our brain tumor classification task, VGG16 can be fine-tuned to learn the specific features of MRI images, leveraging its deep architecture to capture intricate details.

\textbf{ResNet50:} ResNet50 is another pretrained model that has gained popularity for its performance and efficiency. It introduces a residual learning framework to ease the training of deep networks. ResNet50's architecture allows it to maintain performance while being computationally efficient, making it a suitable choice for our dataset.

\paragraph{Deep Learning with U-Net:}
U-Net is a deep learning model specifically designed for biomedical image segmentation. Its architecture consists of a contracting path to capture context and a symmetric expanding path for precise localization. U-Net has been widely used in medical imaging due to its ability to segment images accurately. In the context of brain tumor classification, U-Net can be adapted to highlight and classify different regions of MRI scans, making it a powerful tool for our project.

\paragraph{Model Selection Process:}
The model selection process involves evaluating the suitability of each model based on the dataset characteristics and the specific requirements of the classification task. Given the relatively small size of our dataset (dataset\_19), models that can generalize well from limited data are preferred. 

% \begin{enumerate}
%   \item \textbf{VGG16} is selected for its deep architecture and proven performance in image classification tasks.
%   \item \textbf{ResNet50} is chosen for its balance of high accuracy and computational efficiency.
%   \item \textbf{U-Net} is included due to its specialized design for biomedical image segmentation, which can be adapted for classification purposes.
% \end{enumerate}

By leveraging these models, we aim to achieve high accuracy in classifying brain MRI images, thereby supporting healthcare professionals in the efficient diagnosis of brain tumors and facilitating early intervention. The combination of transfer learning and deep learning techniques ensures that our approach is both robust and effective, despite the limited size of the dataset.

\subsection{Metrics}\label{metrics}

After training the models, their performance is evaluated using a variety of metrics, providing a comprehensive understanding of their effectiveness in classification tasks. Below are the commonly used metrics, along with their mathematical formulations based on True Positive (TP), False Positive (FP), True Negative (TN), and False Negative (FN) values.

The confusion matrix offers a detailed breakdown of the model's predictions, illustrating the number of true positives (TP), true negatives (TN), false positives (FP), and false negatives (FN). This matrix is crucial for identifying the model's strengths and weaknesses in classifying different classes, allowing for a granular analysis of performance.

Precision is a metric that measures the proportion of true positive predictions among all positive predictions made by the model. It is calculated using the formula \(\text{Precision} = \frac{TP}{TP + FP}\). A higher precision value is preferred as it indicates the model's ability to avoid false positives, making it a critical measure in scenarios where the cost of false positives is high.

Recall, also known as sensitivity, measures the proportion of true positive predictions among all actual positive instances in the dataset. It is computed using the formula \(\text{Recall} = \frac{TP}{TP + FN}\). A higher recall value is desirable as it indicates the model's ability to capture all positive instances, which is particularly important in medical diagnoses where missing a positive case can have severe consequences.

The F1 score, which is the harmonic mean of precision and recall, provides a balanced measure of the model's performance by considering both false positives and false negatives. The formula for the F1 score is \(\text{F1 Score} = 2 \times \frac{\text{Precision} \times \text{Recall}}{\text{Precision} + \text{Recall}}\). A higher F1 score signifies better performance, especially in cases where a balance between precision and recall is essential.

Support refers to the number of instances in each class. It helps identify class imbalances and assess the model's performance across different classes, providing context to other metrics.

The Dice Similarity Coefficient (DSC) is a metric commonly used in medical image segmentation tasks. It measures the overlap between the predicted and ground truth segmentation masks, with a higher DSC indicating better segmentation accuracy. The formula for DSC is \(\text{DSC} = \frac{2 \times TP}{2 \times TP + FP + FN}\).

Specificity measures the proportion of true negative predictions among all actual negative instances in the dataset. It is calculated as \(\text{Specificity} = \frac{TN}{TN + FP}\). A higher specificity value is preferred as it indicates the model's ability to avoid false positives, which is crucial in ensuring that negative cases are accurately identified.

Accuracy measures the proportion of correct predictions made by the model across all classes, providing an overall assessment of the model's performance. It is computed using the formula \(\text{Accuracy} = \frac{TP + TN}{TP + TN + FP + FN}\). A higher accuracy value signifies a better-performing model.

Finally, the Receiver Operating Characteristic (ROC) curve is a graphical plot that illustrates the diagnostic ability of a binary classifier system as its discrimination threshold is varied. The Area Under the ROC Curve (AUC) measures the entire two-dimensional area underneath the ROC curve, providing an aggregate measure of performance across all classification thresholds. A higher AUC value is preferred as it indicates better model performance.

These metrics, when analyzed together, offer a comprehensive evaluation framework. Precision, recall, and F1 score primarily assess the model's performance on individual classes, while the confusion matrix and support provide detailed insights into prediction distribution. DSC is particularly crucial for segmentation tasks, ensuring accurate overlap measurement between predicted and actual segmentation masks. Specificity and accuracy offer additional layers of performance evaluation, ensuring robust and reliable model assessment.




%%%%%%%%%%%%%%%%%%%%%%%%%%%%%%%%%%%%%%%%%%%%%%%%%
% Individual Model SUBSECTIONS
%%%%%%%%%%%%%%%%%%%%%%%%%%%%%%%%%%%%%%%%%%%%%%%%%
\import{Sections/models/}{unet.tex}

\import{Sections/models/}{inceptionv3.tex}

\import{Sections/models/}{resnet50.tex}
